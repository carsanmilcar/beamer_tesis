\documentclass[9pt]{beamer}
\usepackage[spanish]{babel}
\usepackage[utf8]{inputenc}
\usepackage{graphicx}
\usepackage{animate}
\usetheme{Madrid}
\beamertemplatenavigationsymbolsempty

% Colores de ejemplo
\definecolor{colorUva}{RGB}{0,107,166}
\definecolor{darkgreen}{rgb}{0.01,0.25,0.04}

% Datos de la presentación (modificar según sea necesario)
\title{Título de la Presentación}
\author{Autor}
\institute{Institución}
\date{\today}

% Pie de página con número total de diapositivas
\setbeamertemplate{footline}{%
  \leavevmode%
  \hbox{%
    \begin{beamercolorbox}[wd=.8\paperwidth,ht=2.25ex,dp=1ex,center]{author in head/foot}%
      \usebeamerfont{author in head/foot}\insertshorttitle
    \end{beamercolorbox}%
    \begin{beamercolorbox}[wd=.2\paperwidth,ht=2.25ex,dp=1ex,right]{date in head/foot}%
      \usebeamerfont{date in head/foot}\insertframenumber{} / \inserttotalframenumber\hspace*{1ex}
    \end{beamercolorbox}%
  }%
  \vskip0pt%
}

\begin{document}

\begin{frame}
  \maketitle
\end{frame}

\begin{frame}{\contentsname}
  \tableofcontents
\end{frame}

\section{Introducción}
\begin{frame}{Introducción}
  Texto de ejemplo para la diapositiva de introducción.
\end{frame}

% --- Ejemplo de inclusión de animaciones ---
\section{Animaciones}
\begin{frame}{Ejemplo de GIF}
  % Coloque las imágenes secuenciales en 'carpeta/nombre-#.ext'
  % y ajuste la velocidad con el primer parámetro (fps)
  % Descomente la línea siguiente y ajuste la ruta de las imágenes para incluir un GIF
  \begin{figure}
\centering
%    \animategraphics[autoplay,loop,width=0.8\textwidth]{5}{carpeta/gif-}{1}{10}
  \end{figure}
\end{frame}

\end{document}
